\begin{abstract}
We present challenges faced in making a domain-specific language (DSL) for graph algorithms adapt to varying requirements to generate a spectrum of efficient parallel codes. Graph algorithms are at the heart of several applications, and achieving high performance with them has become critical due to tremendous growth of irregular data. However, irregular algorithms are quite challenging to parallelize automatically, due to access patterns influenced by the input graph -- which is unavailable until execution. Former research has addressed this issue by designing DSLs for graph algorithms, which restrict generality, but allow efficient code-generation for various backends. Such DSLs are, however, too rigid, and do not adapt to changes in backends or to input graph properties or to both. We narrate our experiences in making an existing DSL, named Falcon, adaptive. The biggest challenge in the process is to not change the DSL code for specifying the algorithm. We illustrate the effectiveness of our proposal by auto-generating codes for vertex-based versus edge-based graph processing, synchronous versus asynchronous execution, and CPU versus GPU backends.
\end{abstract}
